\documentclass[makeidx, a4paper, 14pt]{extarticle}
\usepackage{makeidx}

% Enable cyrillic support
\usepackage[T1,T2A]{fontenc}
\usepackage[utf8]{inputenc}
\usepackage[main=russian,english]{babel}

% Math
\usepackage{amssymb, amsthm, amsmath, euscript}

% Figures
\usepackage{graphicx}
\usepackage{subfigure}

% Title page
\usepackage{fancyhdr}

% Section titles.
\usepackage{titlesec}

% Sections
\newcommand{\sectionbreak}{\clearpage}

% Symbols
\newcommand{\R}{\mathbb{R}}

% Define blocks.
\newtheorem{definition}{Определение}

\begin{document}

\begin{titlepage}
    \thispagestyle{fancy}
    \renewcommand{\headrulewidth}{0pt}
    \begin{center}
        МОСКОВСКИЙ ГОСУДАРСТВЕННЫЙ УНИВЕРСИТЕТ \\
        имени М.В. ЛОМОНОСОВА
        \medskip
        ФИЛИАЛ В ГОРОДЕ ТАШКЕНТЕ

        \bigskip
        \bigskip

        ФАКУЛЬТЕТ ПРИКЛАДНОЙ МАТЕМАТИКИ И ИНФОРМАТИКИ
        \medskip
        КАФЕДРА ПРИКЛАДНОЙ МАТЕМАТИКИ И ИНФОРМАТИКИ

        \bigskip
        \hrule
        \bigskip

        \large{Ванесян Роман Грачикович}

        \bigskip
        \bigskip

        \textbf{ВЫПУСКНАЯ КВАЛИФИКАЦИОННАЯ РАБОТА}

        \bigskip
        \bigskip

        \textbf{«Оптическое распознавание схем из функциональных элементов»}

        \bigskip
        \bigskip
        \bigskip
        \bigskip
        \bigskip
        \bigskip

        \begin{small}
            \begin{flushleft}
              Научный руководитель, \\
              к.ф.-м.н. \underline{\hspace{10.25cm}} Шуткин Ю.С.
            \end{flushleft}
        
            \smallskip
        
            \begin{flushright}
              «\underline{\hspace{1cm}}» \underline{\hspace{3.5cm}} 2020 г.
            \end{flushright}
        \end{small}

        \cfoot{ТАШКЕНТ - 2020}
    \end{center}
\end{titlepage}

\tableofcontents
\newpage

\section{АА}

\begin{definition}
    Изображением будем называть множество \\ ${I = \{p_i \mid p_i \in C\}}$ на плоскости,
    где множество $C$ --- множество любой природы.
\end{definition}

\begin{definition}
    Изображение для которого множество $C$ определено как ${\{0, 1\}}$ будем называть двоичным изображением.
\end{definition}

\begin{definition}
Множество ${V = \{p_i \mid p_i \in D\}}$ будем считать вершиной, если:
\begin{itemize}
    \item Пусть ${F \subset V}$. Подмножество ${H = \{p_i \mid p_i \in F, p_i=1\}}$ образует одну из следующих фигур: треугольник, окружность, либо прямоугольник.
    \item Пусть ${G = F \setminus H}$. Существует такое подмножество \\
          ${M=\{p_i \mid p_i \in G, p_i=1\}}$ --- метка вершины.
    \item Никакие две рядом лежащие вершины не расположены так,
          что пересечение минамальных гиперпрямоугольников (прямоугольник в смысле $\R^2$)
          содержащих соответствующие вершины есть множество не пустое, то есть: \\
          ${\forall V_i, V_j, \quad V_i \neq V_j, i \neq j: P(V_i) \cap P(V_j) = \varnothing}$.
\end{itemize}
\end{definition}

\begin{definition}
Ребром будем называть жорданову дугу образованную последовательностью точек ${p_i=1}$
и соединяющее вершины ${v_i, v_j, i \neq j}$.
\end{definition}

\section{asdasd}

\begin{definition}
    Схемой из функциональных элементов (СФЭ) над базисом ${F \cup X}$ будем называть ориентированны граф,
    каждая вершина которого помечена одним из элементов множества ${F \cup X}$, либо обозначением формулы.
\end{definition}

При изображении схемы из функциональных элементов входы будем обозначать окружностями, внутри которых записаны входные переменные.
Вершины являющиеся операциями, --- треугольниками, внутри которых записаны обозначения соответствующих функций. А вершину обозначающую
выход СФЭ, будем обозначать прямоугольником, внутри которого записано обозначение реализуемой формулы. Выходы функций будем отмечать "выходными" стрелками.

Без ограничения общности будем полагать, что СФЭ определена над стандартным базисом ${\{x_1 \wedge x_2, x_1 \vee x_2, \neg{x_1}\}}$.

Так же будем рассматривать задачу о распознавании двоичного изображения СФЭ.
Так как любую задачу о распознавании изображения СФЭ можно свести к аналогичной задаче о распознавания
двоичного изображения СФЭ, путем введения биективного отображения, которое по какому-то заранее известному
закону каждой точке изображения ставит в соответствие точку в двоичном отображении.

\end{document}
